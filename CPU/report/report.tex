\documentclass{article}
\usepackage{ctex}
\usepackage{graphicx}
\usepackage{amsmath}
\usepackage{indentfirst}
\usepackage{amstext}
\usepackage{amssymb}

\title{体系结构大作业报告}
\author{2016ACM班~~~~范裕达}
\date{\today}
\begin{document}
\maketitle

\section{摘要}
本次体系结构的大作业是写一个RISC-V框架下的CPU,我选择了使用上课所学的Tomasulo算法,并实现了基本的指令集。我的代码都是在 gnu-tool-chain 和 Icarus Verilog version 10.1的框架下进行编译运行,输出的波形可以通过生成的vcd用gtkwave查看。

\section{框架简述}
我的代码使用的经典的issue-execution-commit结构,采用的是乱序执行和顺序写回的策略,对于分支跳转语句采用的是stall的方法。

\subsection{issue}
	issue阶段包括了从内存fetch指令,并且进行解码,从而发射到对应的保留站和ROB里,并且获得该条指令在ROB里的位置,并且使用这个位置来进行renaming。ROB在这个阶段需要把这个指令要占用的寄存器地址发射出去,在regstatus里打上标记。

\subsection{execution}
	execution阶段主要是各保留站在工作,在时钟周期开始的时候,各保留站检查一下站内的指令,如果发现有一条指令ready了,就把它进行运算,并把所得的结果和其在ROB中的位置一同广播出去。一共有两个CDB,一个是给Add运算使用的,一个是给Load运算使用的,对于Store和Bne运算,则不需要广播,直接传输到ROB就行。同时各保留站时刻保持监听CDB总线,如果有广播的数据是自己需要的,就把它进行站内的更新。

\subsection{commit}
	commit阶段基本都是ROB在工作,主要负责各指令的写回。ROB是本框架中唯一一个有权限对寄存器和内存进行修改的元件。对于跳转指令,ROB则要发射信号给pcControl,使其更改pc值。同时对于有寄存器写入的指令,ROB将于regstatus进行交互,以判断是否将对应的占用的标记去除。
	
\section{具体功能介绍}
\subsection{regfile regstatus}
这两个元件主要负责对寄存器的读写,尤其是标示寄存器的占用情况,对于正在被写入的寄存器,它会被打上tag表示正在由哪条指令写入(以ROBindex标示),如果未被占用,则会用invalid来标志。

\subsection{reorderBuffer}
ROB有几大功能:一是应对各个运算元件的数据的request,因为一个寄存器正在被写入而且对应的指令已经不在保留站里了,那么对应的数据一定在ROB里面。二是接收各运算元件的结果,对队列中的指令进行更新。三是写回指令,更重要的是更新pc,并且决定是否进行stall和跳转,具有控制的功能。
\section{不足与困难}
	一是如何保证对内存的操作是有序的,防止RAW的anti-hazard。解决方法有两种,一是对内存操作也开一个buffer,保证有序,二是保证ROB里有Store的时候,Load指令不会上线。
	
	二是如何应对JALR操作,因为该指令还涉及到了对寄存器的写入,在我的框架中我把它扔进了加法保留站,从而对分支预测带来了不小的麻烦。后来经过助教的指点,发现可以直接在Decoder阶段就进行跳转,同时将它翻译成一个Li指令,而不经过ROB的阶段。
\end{document}